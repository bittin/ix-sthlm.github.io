\documentclass[a4paper,11pt,oneside]{article}
\usepackage[swedish]{babel}
\usepackage[utf8]{inputenc}
\usepackage[T1]{fontenc}
\usepackage{fancyhdr}
\usepackage{verbatim}
\usepackage{graphicx}
\usepackage{booktabs}
\pagestyle{fancy}
\fancyfoot{}
\lhead{Konstituerande styrelsemöte}
\rhead{2018--05--07}
\fancyfoot{}
\fancyfoot[LE, RO]{\thepage}

\begin{document}

\section*{Ixs Konstituerande styrelsemöte för år 2018}

\section*{Mötets öppning}
Mötet öppnades 19:10

\section*{Val av mötesordförande}
Elis Hirwing valdes till mötesordförande.

\section*{Val av mötessekreterare}
Michael Stahre valdes till mötessekreterare.

\section*{Val av justeringsperson tillika rösträknare}
Erik Welander valdes till justeringsperson tillika rösträknare.

\section*{Val av kassör}
Erik Welander valdes till kassör.

\section*{Val av firmatecknare}
Elis Hirwing (XXXXXX-XXXX) och Erik Welander (XXXXXX-XXXX) valdes att var för sig teckna föreningens firma.

\section*{Årsmötesprotokollet}
Protokollet är justerat och skall skrivas ut och justeras efter mötet.

\section*{GDPR-hantering}
Det beslutades att vi skall föra årets medlemslista fysiskt i en pärm, med ett papper per medlem där information om hur informationen kommer hanteras finns och kan godkännas när medlemmen skriver på. Elis ska ta fram denna lapp.
Det beslutades att vi skall ta bort våran Drive-mapp.
Martin Arnér utsågs till Dataskyddsombud och åtogs att gå igenom mappen, skriva ut de gamla dokument som skall sparas (såsom gamla årsmötesdokument samt förra årets medlemslista då vår medlemsrapport till Unga Forskare Stockholm görs först i höst), ta bort alla andra dokument och kontakta de som äger dokument med sina privata konton att ta bort dessa dokument.
Det konstaterades att vi ska skriva en integritetspolicy samt ett inventarie över vilka personuppgifter vi lagrar, var vi lagrar dem och varför. Michael Stahre och Elis Hirwing åtog sig att skriva dessa.

\section*{Mötets avslutande}
Mötet avslutades 20:00

\end{document}
